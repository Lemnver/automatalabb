\section{Lexikalanalys}

Eftersom det skulle skapas ett språk för att skanna \textit{fak.c}, valdes att enbart det alfabet
som \textit{fak.c} använder. Detta gjordes genom att titta på \textit{fak.c}.
\\ \textit{pas.l} användes som en mall för att utveckla språket för specifikt \textit{fak.c},
därav de tokens som existerade men var ej nödvändiga för \textit{fak.c} togs bort, och de som
behövdes laddes till. 

\begin{figure}[!h]
    \includegraphics[width=\linewidth,height=3cm]{bilder/fak_c.png}
    \captionof{figure}{fak.c koden}
    \label{fig:fak c}
\end{figure}


\begin{figure}[!h]
    \includegraphics[width=\linewidth,height=3cm]{bilder/fak_l.png}
    \captionof{figure}{fak.l språket}
    \label{fig:fak l}
\end{figure}


\begin{description}

\item[\textit{\#include}] I Figure[\ref{fig:fak l}] är \textit{\#include} ett token för importering av biblotek. 

\item[\textit{{ID}*.h}] med det fördefinerade \textit{ID} på rad 16 i
\ref{fig:fak l}, som säger att alla ord som är konstruerade med \textit{ID} och slutar
med \textit{.h}

\item[\textit{{DIGIT}+}] includerar alla heltal.

\item[\textit{if|else|return|exit}] är alla nyckel ord som uppstår i \textit{fak.c [\ref{fak c}] }
som existerar i programmerings språket c.

\item[\textit{int|char}] är typdeklarationerna som uppstår i \textit{fak.c [\ref{fak c}]}. 

\item[\textit{\\t|\\n}] en token för att känna igen speciella karaktärer i en sträng sekvens.

\item[\texti{rad 37-42}] tokens som beskriver start och slut på gruppsekvenser för dem
operationer som använder sig av klammrar, måsvingar och paranteser.

\item[\textit{rad 45-49}] beskriver tokens för diverse speciella karaktärer, deklarationer av
tillstånd separerare, operatorer för olika operationer samt deklarationer för strängar och
karaktärer.

\item{\textit{{ID}+}r}] för att skapa tokens åt identifierare används ett reguljärt uttryck för alla strängar som
består av enbart \textit{ID}.

\item{\textit{rad 51-52}} för resterande text finns det uttryck som känner igen tomt utrymme och
tecken som är inte inkluderat i språket.


\end{description}
